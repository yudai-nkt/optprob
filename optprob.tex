\documentclass[a4paper,10pt]{article}
\usepackage[T1]{fontenc}
\usepackage[margin=1.25in]{geometry}
\usepackage{hologo}
\usepackage{lmodern}
\usepackage{ltxdockit}
\usepackage{minted}
\usepackage{multicol}
\usepackage{optprob}
\renewcommand*{\optionlistfont}{\ttfamily}
\makeatletter
\def\pTeX{p\kern-.15em\TeX}
\def\upTeX{up\kern-.15em\TeX}
\patchcmd{\ltd@envitem}{\{{\ltxsyntaxfont\ltxsyntaxlabelfont#1}\}}{{\ltxsyntaxfont\ltxsyntaxlabelfont\{#1\}}}{}{}
\DeclareMathOperator*\maximize{maximize}
\setminted{autogobble,xleftmargin=.5em,tabsize=4}
\makeatother
\title{The \sty{optprob} package\footnote{\url{https://github.com/yudai-nkt/optprob}}}
\author{Yudai NAKATA}
\date{\today}
%
\begin{document}
\maketitle
\begin{abstract}
    \sty{optprob} is a package that offers you useful features for writing optimization problems in a structured manner.
    The package also provides highly customizable options for the appearance of the optimization problems.
\end{abstract}
\tableofcontents
\section{Introduction}
\subsection{Motivation}
In order to illustrate optimization problems, one can utilize \env{alignat} environment from the \sty{amsmath} package.
A simple example like this,

\begin{minted}{latex}
    \documentclass{article}
    \usepakcge{amsmath}
    \DeclareMathOperator*\maximize{maximize}
    \begin{document}
        \begin{alignat}{2}
            & \mathord{\maximize_{x_1,x_2\in\mathbf{R}}} & \quad & c_1x_1+c_2x_2 \\
            & \text{subject to} & & a_{11}x_1+a_{12}x_2 \leq b_1 \\
            & & & a_{21}x_1+a_{22}x_2\leq b_2.
        \end{alignat}
    \end{document}
\end{minted}
will produce
\begin{alignat}{2}
    & \mathord{\maximize_{x_1,x_2\in\mathbf{R}}} & \quad & c_1x_1+c_2x_2 \\
    & \text{subject to} & & a_{11}x_1+a_{12}x_2 \leq b_1 \\
    & & & a_{21}x_1+a_{22}x_2\leq b_2.
\end{alignat}
One can also use other \env{align}-variant environments or \env{array} environment.

Each method has its pros and cons, but none of them are very friendly when it comes to denoting optimization problem.
Writing disgusting numbers of ampersands in every line for the proper alignment is really annoying (to me at least) and could sometimes look ugly if we only input inadequate ampersands.
Also, it is difficult to intuitively grasp the role of each formulae at a glance upon the source code if we put them manually.

\subsection{Objective of this package}
The \sty{optprob} package provides an elegant and semantic syntax for optimization problems.
With the aid of this package, you are free from the concern about the number of ampersands.
You can easily switch the spacing or other settings via $\prm{key} = \prm{value}$ syntax.
Semantically-named macros enhance your code's readability compared to the bare \LaTeX\ expression.

\section{Requirements}
The \sty{optprob} package is supported on the following engines and format.
\begin{itemize}
    \item {\TeX} engine: \TeX, \hologo{pdfTeX}, \hologo{XeTeX}, \hologo{LuaTeX}, {\pTeX} and \upTeX
    \item {\TeX} format: \LaTeXe\ (\hologo{plainTeX} and \hologo{ConTeXt} are not supported.)
\end{itemize}
This package requires \sty{mathtools} and \sty{pgfkeys} to make use of it, and listed below are necessary packages to typeset this documentation besides the prerequisite packages:
\begin{multicols}{3}
\begin{itemize}
    \item \sty{fontenc}
    \item \sty{geometry}
    \item \sty{hologo}
    \item \sty{lmodern}
    \item \sty{ltxdockit}
    \item \sty{minted}
    \item \sty{multicol}
\end{itemize}
\end{multicols}
All the packages above are bundled with recent \TeX~Live by default.
If any of them are not installed in your {\TeX} system, you can download them via CTAN.
\section{Installation}
Clone the repository or download from the Releases.
It includes three files, i.e.\ \file{optprob.pdf}, \file{optprob.sty} and \file{optprob.tex}.
Following TDS ({\TeX} Directory Structure), move each file to the corresponding directories
as follows:
\begin{itemize}
    \item \file{optprob.pdf} and \file{optprob.tex} $\to$ \file{texmf/doc/latex/optprob/}
    \item \file{optprob.sty} $\to$ \file{texmf/tex/latex/optprob/},
\end{itemize}
where \file{texmf} is either \file{\$TEXMFLOCAL} or \file{\$TEXMFHOME}.
Make sure to run \mbox{\texttt{mktexlsr}} if necessary.

\section{Usage}
\subsection{Package loading}
Just put this line in your preamble:
\begin{minted}{latex}
\usepackage{optprob}
\end{minted}
For the time being, this package has no options available.

\subsection{Macros and environments}
\begin{ltxsyntax}
    \cmditem{argmax}\par
    prints ``$\argmax$'', which denotes arguments of the maximum.

    \cmditem{argmin}\par
    is the minimum counterpart of \texttt{\string\argmax}.

    \envitem{optimize}[layout formatting]{operation type}\par
    either \mbox{\verbatimfont max} or \mbox{\verbatimfont min} can be used for \prm{operation type} according to the problem you want to represent.
    You can specify the design of the environment in the \prm{layout formatting} option.
    Each setting can be given in $\prm{key}=\prm{value}$ syntax as follows.

    \begin{optionlist}
        \boolitem[false]{abbrev}
        to or not to abbreviate the terms such as maximize, minimize and subject~to.

        \boolitem[true]{showeqnum}
        to or not to show the equation numbers in each line.

        \optitem[1em]{space}{\prm{dimension}}
        space between two columns.
    \end{optionlist}

    \envitem{maximize}[layout formatting]\par
    is equivalent to the \env{optimize} environment with the mandatory argument set to \mbox{\verbatimfont max}.

    \envitem{minimize}[layout formatting]\par
    is equivalent to the \env{optimize} environment with the mandatory argument set to \mbox{\verbatimfont min}.
    Note that this and \env{maximize} environments are defined only if \cs{minimize} and \cs{maximize} are not defined in the preamble respectively.
\end{ltxsyntax}

Within these environments above, the following macros are locally defined:
\begin{ltxsyntax}
    \cmditem{objfunc}[variables]{objective function}\par
    sets the objective function to be maximized or minimized and denotes the variables over which the objective function is optimized.
    The default value for optional argument is an empty string.
    You \emph{must} use this macro once per one environment.

    \cmditem{addconstraint}{constraint}\par
    adds constraints of the problem.
    You can use this macro as many times as necessary.
\end{ltxsyntax}
\newpage
\subsection{Demos}
I will  demonstrate some simple examples.

\begin{multicols}{2}
    \begin{minted}{latex}
        \begin{optimize}{max}
            \objfunc[x, y]{2x-y}
            \addconstraint{x-y\le1}
            \addconstraint{-x+y\le-2}
            \addconstraint{x, y\ge0}
        \end{optimize}
    \end{minted}
    \begin{optimize}{max}
        \objfunc[x, y]{2x-y}
        \addconstraint{x-y\le1}
        \addconstraint{-x+y\le-2}
        \addconstraint{x, y\ge0}
    \end{optimize}
\end{multicols}
\begin{multicols}{2}
\begin{minted}{latex}
\begin{minimize}[
    abbrev=true,space=2em,showeqnum=false
]
    \objfunc{\sin\theta+2\cos\phi}
    \addconstraint{3\theta-\phi=7\pi}
\end{minimize}
\end{minted}
\begin{minimize}[abbrev=true,space=2em,showeqnum=false]
    \objfunc{\sin\theta+2\cos\phi}
    \addconstraint{3\theta-\phi=7\pi}
\end{minimize}
\end{multicols}

\section{Acknowledgements}
The author is thankful to those who made and/or have been mantaining the packages on which \sty{optprob} has dependency.

\section{License}
This package is distributed under the MIT License: \url{https://opensource.org/licenses/MIT}.

\section{Changelog}
\paragraph{v0.2.0} (July 5, 2016)
\begin{itemize}
    \item Add an option for switching the appearance of equation numbers.
    \item Change the way of specifying variables in \env{optimize} environment and friends.
\end{itemize}
\paragraph{v0.1.1} (July 5, 2016)
\begin{itemize}
    \item Fix the incompatibility with \env{minipage} environment.
\end{itemize}
\paragraph{v0.1.0} (May 25, 2016)
\begin{itemize}
    \item Fisrt publication.
\end{itemize}
\end{document}
